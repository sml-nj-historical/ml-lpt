%!TEX root = manual.tex
%
\chapter{Change history}
\label{ch:history}

Here is a history of changes to the SML/NJ Language Processing Tools.
More details can be found in the SML/NJ \texttt{NOTES} and \texttt{README} files.
\begin{description}
  \item[SML/NJ 110.78 (not yet released)]
    \mbox{}\\[0.5em]
    Added \texttt{\%default} directive to allow non-nullary tokens to be inserted as
    an error-repair action.
    \\[0.5em]
    Improved error messages for the situation where the lexer specification has an unclosed string.
  \item[SML/NJ 110.77]
    \mbox{}\\[0.5em]
    Fixed an inconsistency in the way that \mlantlr{} and \ulex{} handled the contents of
    a \texttt{\%defs} declaration.  \ulex{} made these definitions visible in the \texttt{UserDeclarations}
    substructure, whereas \mlantlr{} hid them.  We have changed the behavior of ml-ulex to match
    that of \mlantlr{} (\ie{}, hide the user definitions).  We chose to hide the user definitions
    in ml-ulex because they are usually not useful outside the lexer, hiding them reduces The
    size of the generated code, and definitions that are needed outside the lexer can be
    defined in an external module.  Note that the \texttt{UserDeclarations} substructure remains
    visible when \ulex{} is run in \texttt{ml-lex} compatibility mode.
    \\[0.5em]
    Added the \texttt{actionToString'} and \texttt{repairToString'} functions
    to the \texttt{AntlrRepair} structure.  These functions allow one to
    specialize the printing of tokens based on whether they are being added or deleted.
    \\[0.5em]
    Removed the \texttt{toksToString} function from the tokens structure that ml-antlr
    generates.  It was originally for use by the \texttt{AntlrRepair} structure, but that
    structure does not use it.
  \item[SML/NJ 110.72]
    \mbox{}\\[0.5em]
    Added \texttt{--strict-sml} flag to \ulex{} for MLton compatibility.
    \\[0.5em]
    Added \texttt{\%header} directive to the \mlantlr{} parser generator.
\end{description}%
