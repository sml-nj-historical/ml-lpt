\chapter[\ulex]{Usage: \ulex}

\section{Overview}

\section{Specification format}

\Grammar{
\GFirstB{spec}
	{$($ declaration \T{;} $)^*$}

\GFirstB{declaration}
	{directive}
\GNextB
	{rule}

\GFirstB{directive}
	{\kw{charset} $($ \T{ASCII7} $|$ \T{ASCII8} $|$ \T{UTF8} $)$}
\GNextB
	{\kw{defs} code}
\GNextB
	{\kw{let} ID \T{=} re}
\GNextB
	{\kw{name} ID}
\GNextB
	{\kw{states} ID$^+$}
	
\GFirstB{code}
	{ \T{(} $\dots$ \T{)} }
	
\GFirstB{rule}
	{re \T{=>} code}
	
\GFirstB{re}
	{{\rm any nonreserved, nonwhitespace character or escape code}}
\GNextB
	{re \T{*}}
\GNextB
	{re \T{?}}
\GNextB
	{re \T{+}}
\GNextB
	{\T{\^{ }} re}
\GNextB
	{re \T{|} re}
\GNextB
	{re \T{\&} re}
\GNextB
	{re \T{/} re}
\GNextB
	{re \T{\$}}
\GNextB
	{STRING}
\GNextB
	{\T{\{} ID \T{\}}}
\GNextB
	{\T{[} \T{\^{ }}$^?$ $($ char \T{-} char $|$ char $)^+$ \T{]}}
\GNextB
	{re re}
\GNextB
	{\T{(} re \T{)}}
\GNextB
	{\T{.}}
\GNextB
	{\T{\_}}
}

\section{An example}

\begin{verbatim}
calc.lex

%name CalcLex;
%charset UTF8;

(* note: number and letter are predefined unicode 
 * character classes.
 *)
%let int = [:number:]*;
%let id = [:letter:]([:letter:] | [:number:])*;

%defs (
  open CalcParse.Tok
);

let     => ( KW_let );
in      => ( KW_in );
{id}    => ( ID (yytext()) );
{int}   => ( NUM (valOf (Int.fromString (yytext()))) );
"="     => ( EQ );
"+"     => ( PLUS );
"-"     => ( MINUS );
"*"     => ( TIMES );
"("     => ( LP );
")"     => ( RP );
[:whitespace:]
        => ( yyignore() );
.       => ( (* handle error *) );
\end{verbatim}
